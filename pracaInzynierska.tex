\documentclass[a4paper, titlepage]{article}

\usepackage{polski}
\usepackage[utf8]{inputenc}
\usepackage{amsmath}
\usepackage{graphicx}
\usepackage{tabularx}
\usepackage{listings}
\usepackage{indentfirst}
\usepackage{caption}
\usepackage{float}
\usepackage{url}
\usepackage{dirtree}


\DeclareCaptionFont{white}{\color{white}}

\DeclareCaptionFormat{listing}{%

	\parbox{\textwidth}{\colorbox{gray}{\parbox{\textwidth}{#1#2#3}}\vskip-4pt}}

	\captionsetup[lstlisting]{format=listing,labelfont=white,textfont=white}

	\lstset{frame=lrb,xleftmargin=\fboxsep,xrightmargin=-\fboxsep}

	\setcounter{page}{0}

	\begin{document}

	\author{Mateusz Olczak}

	\title{Webowa aplikacja do przeprowadzania testów - Praca Inżynierska}
	
	\maketitle
	\newpage


	\section{\textbf{Wstęp}}

	W dzisiejszych czasach, dzięki nowym technologiom oraz rosnących przepustowościach pasma internetowego dostępnego dla zwykłego użytkownika, coraz częściej odchodzi się od projektowania aplikacji desktopowych na rzecz aplikacji webowych. Aplikacje webowe są w pełni niezależne od systemu operacyjnego, zazwyczaj cechują się o wiele mniejszymi wymaganiami sprzętowymi (ponieważ przeważająca część logiki, wykonywana jest po stronie serwera). Ponadto dzięki całej gamie frameworków, pomagających w tworzeniu widokowej wartswy aplikacji, systemy tworzone jako aplikacje webowe, są o wiele bardziej estetyczne, ergonomiczne dla użytkownika, a dla twórcy oprogramowania, łatwiejsze do modyfikacji. 
	Jak widać, istnieje wiele powodów dla których warto porzucić aplikacje desktopowe na rzecz przenośnych systemów przeglądarkowych. Dlatego też realizowana przezemnie aplikacja, reprezentuje nowy nurt.
	

	\subsection{Cele}
	Głównym założeniem projektu było stworzenie aplikacji webowej, której zadaniem było umożliwienie tworzenia pytań wyboru oraz generowanie testów z wcześniej utworzonych pytań.
	Ponadto, dużą uwagę nałożono na bezpieczeństwo oraz responsywność systemu. Aplikacja miała być intuicyjna, aby każdy student mógł rozwiązać test, bez konieczności wcześniejszego wprowadzenia do korzystania z systemu. 
	\subsection{Zakres pracy}
	W mojej pracy zawarłem podstawowe informacje o działaniu systemu, opis funkcjonalności, wykorzystanych technologii oraz instrukcję korzystania z systemu.
	
	\section{\textbf{Teoria}}
	\subsection{Spis treści}
	\subsection{Prawa autorskie}
	\subsection{Odpowiedzialność dyplomanta}
	
\end{document}

